% Options for packages loaded elsewhere
\PassOptionsToPackage{unicode}{hyperref}
\PassOptionsToPackage{hyphens}{url}
%
\documentclass[
]{article}
\usepackage{amsmath,amssymb}
\usepackage{iftex}
\ifPDFTeX
  \usepackage[T1]{fontenc}
  \usepackage[utf8]{inputenc}
  \usepackage{textcomp} % provide euro and other symbols
\else % if luatex or xetex
  \usepackage{unicode-math} % this also loads fontspec
  \defaultfontfeatures{Scale=MatchLowercase}
  \defaultfontfeatures[\rmfamily]{Ligatures=TeX,Scale=1}
\fi
\usepackage{lmodern}
\ifPDFTeX\else
  % xetex/luatex font selection
\fi
% Use upquote if available, for straight quotes in verbatim environments
\IfFileExists{upquote.sty}{\usepackage{upquote}}{}
\IfFileExists{microtype.sty}{% use microtype if available
  \usepackage[]{microtype}
  \UseMicrotypeSet[protrusion]{basicmath} % disable protrusion for tt fonts
}{}
\makeatletter
\@ifundefined{KOMAClassName}{% if non-KOMA class
  \IfFileExists{parskip.sty}{%
    \usepackage{parskip}
  }{% else
    \setlength{\parindent}{0pt}
    \setlength{\parskip}{6pt plus 2pt minus 1pt}}
}{% if KOMA class
  \KOMAoptions{parskip=half}}
\makeatother
\usepackage{xcolor}
\usepackage[margin=1in]{geometry}
\usepackage{color}
\usepackage{fancyvrb}
\newcommand{\VerbBar}{|}
\newcommand{\VERB}{\Verb[commandchars=\\\{\}]}
\DefineVerbatimEnvironment{Highlighting}{Verbatim}{commandchars=\\\{\}}
% Add ',fontsize=\small' for more characters per line
\usepackage{framed}
\definecolor{shadecolor}{RGB}{248,248,248}
\newenvironment{Shaded}{\begin{snugshade}}{\end{snugshade}}
\newcommand{\AlertTok}[1]{\textcolor[rgb]{0.94,0.16,0.16}{#1}}
\newcommand{\AnnotationTok}[1]{\textcolor[rgb]{0.56,0.35,0.01}{\textbf{\textit{#1}}}}
\newcommand{\AttributeTok}[1]{\textcolor[rgb]{0.13,0.29,0.53}{#1}}
\newcommand{\BaseNTok}[1]{\textcolor[rgb]{0.00,0.00,0.81}{#1}}
\newcommand{\BuiltInTok}[1]{#1}
\newcommand{\CharTok}[1]{\textcolor[rgb]{0.31,0.60,0.02}{#1}}
\newcommand{\CommentTok}[1]{\textcolor[rgb]{0.56,0.35,0.01}{\textit{#1}}}
\newcommand{\CommentVarTok}[1]{\textcolor[rgb]{0.56,0.35,0.01}{\textbf{\textit{#1}}}}
\newcommand{\ConstantTok}[1]{\textcolor[rgb]{0.56,0.35,0.01}{#1}}
\newcommand{\ControlFlowTok}[1]{\textcolor[rgb]{0.13,0.29,0.53}{\textbf{#1}}}
\newcommand{\DataTypeTok}[1]{\textcolor[rgb]{0.13,0.29,0.53}{#1}}
\newcommand{\DecValTok}[1]{\textcolor[rgb]{0.00,0.00,0.81}{#1}}
\newcommand{\DocumentationTok}[1]{\textcolor[rgb]{0.56,0.35,0.01}{\textbf{\textit{#1}}}}
\newcommand{\ErrorTok}[1]{\textcolor[rgb]{0.64,0.00,0.00}{\textbf{#1}}}
\newcommand{\ExtensionTok}[1]{#1}
\newcommand{\FloatTok}[1]{\textcolor[rgb]{0.00,0.00,0.81}{#1}}
\newcommand{\FunctionTok}[1]{\textcolor[rgb]{0.13,0.29,0.53}{\textbf{#1}}}
\newcommand{\ImportTok}[1]{#1}
\newcommand{\InformationTok}[1]{\textcolor[rgb]{0.56,0.35,0.01}{\textbf{\textit{#1}}}}
\newcommand{\KeywordTok}[1]{\textcolor[rgb]{0.13,0.29,0.53}{\textbf{#1}}}
\newcommand{\NormalTok}[1]{#1}
\newcommand{\OperatorTok}[1]{\textcolor[rgb]{0.81,0.36,0.00}{\textbf{#1}}}
\newcommand{\OtherTok}[1]{\textcolor[rgb]{0.56,0.35,0.01}{#1}}
\newcommand{\PreprocessorTok}[1]{\textcolor[rgb]{0.56,0.35,0.01}{\textit{#1}}}
\newcommand{\RegionMarkerTok}[1]{#1}
\newcommand{\SpecialCharTok}[1]{\textcolor[rgb]{0.81,0.36,0.00}{\textbf{#1}}}
\newcommand{\SpecialStringTok}[1]{\textcolor[rgb]{0.31,0.60,0.02}{#1}}
\newcommand{\StringTok}[1]{\textcolor[rgb]{0.31,0.60,0.02}{#1}}
\newcommand{\VariableTok}[1]{\textcolor[rgb]{0.00,0.00,0.00}{#1}}
\newcommand{\VerbatimStringTok}[1]{\textcolor[rgb]{0.31,0.60,0.02}{#1}}
\newcommand{\WarningTok}[1]{\textcolor[rgb]{0.56,0.35,0.01}{\textbf{\textit{#1}}}}
\usepackage{graphicx}
\makeatletter
\def\maxwidth{\ifdim\Gin@nat@width>\linewidth\linewidth\else\Gin@nat@width\fi}
\def\maxheight{\ifdim\Gin@nat@height>\textheight\textheight\else\Gin@nat@height\fi}
\makeatother
% Scale images if necessary, so that they will not overflow the page
% margins by default, and it is still possible to overwrite the defaults
% using explicit options in \includegraphics[width, height, ...]{}
\setkeys{Gin}{width=\maxwidth,height=\maxheight,keepaspectratio}
% Set default figure placement to htbp
\makeatletter
\def\fps@figure{htbp}
\makeatother
\setlength{\emergencystretch}{3em} % prevent overfull lines
\providecommand{\tightlist}{%
  \setlength{\itemsep}{0pt}\setlength{\parskip}{0pt}}
\setcounter{secnumdepth}{-\maxdimen} % remove section numbering
\ifLuaTeX
  \usepackage{selnolig}  % disable illegal ligatures
\fi
\IfFileExists{bookmark.sty}{\usepackage{bookmark}}{\usepackage{hyperref}}
\IfFileExists{xurl.sty}{\usepackage{xurl}}{} % add URL line breaks if available
\urlstyle{same}
\hypersetup{
  pdftitle={midterm\_review.R},
  pdfauthor={kyle},
  hidelinks,
  pdfcreator={LaTeX via pandoc}}

\title{midterm\_review.R}
\author{kyle}
\date{2024-03-27}

\begin{document}
\maketitle

\begin{Shaded}
\begin{Highlighting}[]
\CommentTok{\# Common Data Types in R}

\CommentTok{\# \textless{}{-} is the assignment operator}

\NormalTok{int }\OtherTok{\textless{}{-}} \DecValTok{8}
\NormalTok{float }\OtherTok{\textless{}{-}} \FloatTok{5.40} 
\NormalTok{arr }\OtherTok{\textless{}{-}} \FunctionTok{c}\NormalTok{(}\DecValTok{5}\NormalTok{, }\DecValTok{3}\NormalTok{, }\DecValTok{30}\NormalTok{)}
\NormalTok{string }\OtherTok{\textless{}{-}} \StringTok{"Hello World!"}
\NormalTok{booleanTrue }\OtherTok{\textless{}{-}} \ConstantTok{TRUE}
\NormalTok{booleanFalse }\OtherTok{\textless{}{-}} \ConstantTok{FALSE}

\CommentTok{\# Arrays can have different datatypes in R}
\NormalTok{mixedArr }\OtherTok{\textless{}{-}} \FunctionTok{c}\NormalTok{(}\StringTok{"Golf"}\NormalTok{, }\FloatTok{6.4}\NormalTok{, }\DecValTok{97}\NormalTok{) }\CommentTok{\# gets converted to char*}

\CommentTok{\# Operators in R}
\NormalTok{a }\OtherTok{\textless{}{-}} \DecValTok{54}
\NormalTok{b }\OtherTok{\textless{}{-}} \DecValTok{7}

\CommentTok{\# Basic Operations}
\NormalTok{addition }\OtherTok{\textless{}{-}}\NormalTok{ a }\SpecialCharTok{+}\NormalTok{ b}
\NormalTok{subtraction }\OtherTok{\textless{}{-}}\NormalTok{ a }\SpecialCharTok{{-}}\NormalTok{ b}
\NormalTok{multiplication }\OtherTok{\textless{}{-}}\NormalTok{ a }\SpecialCharTok{*}\NormalTok{ b}
\NormalTok{division }\OtherTok{\textless{}{-}}\NormalTok{ a }\SpecialCharTok{/}\NormalTok{ b}
\NormalTok{exponetial }\OtherTok{\textless{}{-}}\NormalTok{  b }\SpecialCharTok{\^{}}\NormalTok{ a}

\CommentTok{\# Logarithms, Exponential, Absolute, and Square Root}

\CommentTok{\# Natural Log}
\NormalTok{logE }\OtherTok{\textless{}{-}} \FunctionTok{log}\NormalTok{(}\DecValTok{3}\NormalTok{)}
\CommentTok{\# Log Base 2}
\NormalTok{logOf2 }\OtherTok{\textless{}{-}} \FunctionTok{log2}\NormalTok{(}\DecValTok{8}\NormalTok{)}
\CommentTok{\# Log Base 10}
\NormalTok{logOf10 }\OtherTok{\textless{}{-}} \FunctionTok{log10}\NormalTok{(}\DecValTok{1000}\NormalTok{)}
\CommentTok{\# Exponential}
\NormalTok{exponent }\OtherTok{\textless{}{-}} \FunctionTok{exp}\NormalTok{(}\DecValTok{9}\NormalTok{)}
\CommentTok{\# Absolute Value}
\NormalTok{absolute }\OtherTok{\textless{}{-}} \FunctionTok{abs}\NormalTok{(}\SpecialCharTok{{-}}\DecValTok{69}\NormalTok{)}
\CommentTok{\# Square Root}
\NormalTok{squareRoot }\OtherTok{\textless{}{-}} \FunctionTok{sqrt}\NormalTok{(}\DecValTok{100}\NormalTok{)}

\CommentTok{\# Vectorized Functions {-} Most Functions in R take a Vector as Input}
\NormalTok{sumArr }\OtherTok{\textless{}{-}} \FunctionTok{sum}\NormalTok{(arr)}

\CommentTok{\# Operator on a Vector}
\NormalTok{power2Vec }\OtherTok{\textless{}{-}}\NormalTok{ arr }\SpecialCharTok{\^{}} \DecValTok{2}

\CommentTok{\# Logical Operators can be applied too!}
\NormalTok{result }\OtherTok{\textless{}{-}}\NormalTok{ sumArr }\SpecialCharTok{\&}\NormalTok{ arr}

\CommentTok{\# This below code will Not Run Because mixedArr is of mixed types}
\CommentTok{\# result2 \textless{}{-} sumArr \& mixedArr}

\CommentTok{\# Indexes start at \textquotesingle{}1\textquotesingle{} in R}
\NormalTok{arr[}\DecValTok{1}\SpecialCharTok{:}\DecValTok{1}\NormalTok{] }\CommentTok{\# Will print the first number in the array which is \textquotesingle{}5\textquotesingle{}}
\end{Highlighting}
\end{Shaded}

\begin{verbatim}
## [1] 5
\end{verbatim}

\begin{Shaded}
\begin{Highlighting}[]
\CommentTok{\# Calculate the following of the Vector [2, 4 ,3]}
\NormalTok{classWork1 }\OtherTok{\textless{}{-}} \FunctionTok{c}\NormalTok{(}\DecValTok{2}\NormalTok{, }\DecValTok{4}\NormalTok{, }\DecValTok{3}\NormalTok{)}
\CommentTok{\# Mean}
\NormalTok{meanWk1 }\OtherTok{\textless{}{-}} \FunctionTok{mean}\NormalTok{(classWork1)}
\CommentTok{\# Magnitude}
\NormalTok{squareWk1 }\OtherTok{\textless{}{-}}\NormalTok{ classWork1}\SpecialCharTok{\^{}}\DecValTok{2}
\NormalTok{sumWk1 }\OtherTok{\textless{}{-}} \FunctionTok{sum}\NormalTok{(squareWk1)}
\NormalTok{mag }\OtherTok{\textless{}{-}} \FunctionTok{sqrt}\NormalTok{(sumWk1) }
\NormalTok{normalizedVec }\OtherTok{\textless{}{-}}\NormalTok{ classWork1 }\SpecialCharTok{/}\NormalTok{ mag}
\NormalTok{magNormalizedVector }\OtherTok{\textless{}{-}} \FunctionTok{sqrt}\NormalTok{(}\FunctionTok{sum}\NormalTok{(normalizedVec}\SpecialCharTok{\^{}}\DecValTok{2}\NormalTok{))}

\CommentTok{\# Mean of Numbers 1 {-} 100}
\NormalTok{mean100 }\OtherTok{\textless{}{-}} \FunctionTok{mean}\NormalTok{(}\FunctionTok{c}\NormalTok{(}\DecValTok{1}\SpecialCharTok{:}\DecValTok{100}\NormalTok{))}

\CommentTok{\# A List in R is a collection of Arbitrary Data Types {-} the right way to have an }
\CommentTok{\# array of mixed type}
\NormalTok{listArr }\OtherTok{\textless{}{-}} \FunctionTok{list}\NormalTok{(}\ConstantTok{TRUE}\NormalTok{,}\StringTok{"Golf"}\NormalTok{, }\FloatTok{6.4}\NormalTok{, }\DecValTok{97}\NormalTok{)}

\CommentTok{\# List Elements can have names}
\NormalTok{listArr }\OtherTok{\textless{}{-}} \FunctionTok{list}\NormalTok{(}\AttributeTok{bool=}\ConstantTok{TRUE}\NormalTok{,}\AttributeTok{string=}\StringTok{"Golf"}\NormalTok{, }\AttributeTok{flt=}\FloatTok{6.4}\NormalTok{, }\AttributeTok{num=}\DecValTok{97}\NormalTok{)}

\NormalTok{listArr[}\StringTok{\textquotesingle{}bool\textquotesingle{}}\NormalTok{] }\CommentTok{\# Will print \textquotesingle{}TRUE\textquotesingle{}}
\end{Highlighting}
\end{Shaded}

\begin{verbatim}
## $bool
## [1] TRUE
\end{verbatim}

\begin{Shaded}
\begin{Highlighting}[]
\CommentTok{\# Another way to access list elements with Names}
\NormalTok{listArr}\SpecialCharTok{$}\NormalTok{num }\CommentTok{\# Will print \textquotesingle{}97\textquotesingle{}}
\end{Highlighting}
\end{Shaded}

\begin{verbatim}
## [1] 97
\end{verbatim}

\begin{Shaded}
\begin{Highlighting}[]
\CommentTok{\# There are also matricies, but more specifically dataframes are used in R}
\end{Highlighting}
\end{Shaded}


\end{document}
